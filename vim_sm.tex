\newcommand{\nbsp}{\vspace{5mm}}
\newcommand{\suchthat}{\mid}
\newcommand{\eqc}{\overline}	% TODO insert ()'s automagically?
%\newcommand{\eqc}[1]{\overline(#1)}
\newcommand{\inv}[1]{#1^{-1}}
\newcommand{\fxnto}{\rightarrow}
\newcommand{\pp}{\varphi}	% random phi symbol
\newcommand{\pstar}{\pp^*}	% p upper star
\newcommand{\of}{\circ}		% fxn composition
\newcommand{\zero}{\mathbb{0}}
\newcommand{\one}{\mathbb{1}}
\newcommand{\intersect}{\cap}
\newcommand{\union}{\cup}

\newcommand{\N}{\mathbb{N}}
\newcommand{\Nplus}{\N^{> 0}}

\newcommand{\Z}{\mathbb{Z}}
\newcommand{\Zplus}{\Z^{\ge 0}}
%\newcommand{\ZZ}{\Z \times \Z}
\newcommand{\ZZ}{\Z^2}

\newcommand{\Q}{\mathbb{Q}}

\newcommand{\R}{\mathbb{R}}
%\newcommand{\RR}{\R \times \R}
\newcommand{\RR}{\R^2}

\newcommand{\Zpx}{\Z_p^\times}

\newcommand{\vectI}[1]{(#1_1, ..., #1_n)}
\newcommand{\vectII}[2]{(#1_1 + #2_1, ..., #1_n + #2_n)}
\newcommand{\vectIII}[3]{(#1_1 + #2_1 + #3_1, ..., #1_n + #2_n + #3_n)}
  
\documentclass[12pt]{article}
\usepackage{amsmath}
\usepackage{amsfonts}
\usepackage{amsthm}
\usepackage{fullpage}
%\usepackage{graphicx}

%use these for 12pt
%\addtolength{\oddsidemargin}{-.875in}
%\addtolength{\evensidemargin}{-.875in}
%\addtolength{\textwidth}{1.75in}
%\addtolength{\topmargin}{-.875in}
%\addtolength{\textheight}{1.75in}
%end 12pt


%\addtolength{\oddsidemargin}{-1in}
%\addtolength{\evensidemargin}{-1in}
%\addtolength{\textwidth}{2in}
%\addtolength{\topmargin}{-1in}
%\addtolength{\textheight}{2in}


\title{Linux and Vim}
\author{Skill Module}
\date{}

\begin{document}
\maketitle

\section{Introduction}
%history of vim
%GNU/linux

\subsection{Why Learn Vim?}
With powerful IDEs like Eclipse and Netbeans available, you might ask why
one should even bother learning vim; it's an archaic text editor lacking
many of the features that make IDEs so palatable. While to some extent
this is true, vim is still an extremely powerful text editor that will
make you much more efficient. Vim has an extremely rich set of
functionality that will allow you to edit text in ways that conventional
text editors do not allow.

If you are a computer science major, in many of your classes you will
need to work in a terminal environment and learning to use vim will be
crucial in allowing you to program.
%%add more stuff here%%

\section{Getting Vim}
Ideally, you should be working with vim in Linux environment. 

\subsection{Linux}
If you work in a linux environment, obtain vim from your package manager.
For instance, on Ubuntu you would do:
$$\text{\texttt{sudo apt-get install vim}}$$
You can also obtain the graphical version of vim by doing:
$$\text{\texttt{sudo apt-get install gvim}}$$

\subsection{Windows}
Follow this link to obtain \texttt{Gvim}. \texttt{Gvim} is a graphical
version of vim: %TODO insert link

\subsection{Mac}
I have no idea %TODO find out how to get vim on macs

\section{Understanding Vim Configuration Files}
{\bf Note: this only applies if you develop on a *nix environment,
which I highly suggest that you do.}
Vim is an extremely configurable text editor. You can change how it 
behaves with regards to colorschemes, syntax coloring, tab spacing,
etc. To deal with all of this, Vim uses a \emph{configuration file}.
This configuration file is named \texttt{.vimrc}. On linux systems, files
whose names are prefixed by dots are hidden. This means that running \texttt{ls}
or just looking through your file explorer, they will not be shown by default.

Vim knows to look for this configuration file in a few places. The location 
it looks first is in your home directory, i.e. \texttt{~/.vimrc}. I won't
go into detail here about actually writing your own configuration files;
for now, copy mine and as you use vim more you can change it suit
your own purposes. I've included the file on the website. %TODO where to put file
Download the file and place it in your home directory after you install vim.

\section{Modes}
If you have already tried opening vim and editing a file, you know that it's
not as intuitive as notepad. Luckily, its also not as bad. Let's
delve into how exactly one can use vim. 

There are two modes in vim, {\bf Command Mode} and {\bf Insert Mode}.
{\bf Insert Mode} is what you'd expect; you can type normally just
like any other text editor. {\bf Command Mode}, or the mode in which
Vim normally begins, is the mode where you can run different commands
to edit the text. 

\subsection{Insert Mode}
%%not much to explain
\subsection{Command Mode}
%%add in all the commands possible here

\section{Examples}
%include examples of how to use vim

\section{Conclusion}
As you have seen, Vim is an extremely powerful text editing tool. {\bf If
you read nothing else, read this.} Learning Vim is incremental. No one 
knows all of Vim, and you certainly don't need to know all of Vim to
be effective using it. {\bf Using Vim can be annoying at first.} At first
you'll feel inefficien

\section{Questions}
%ask questions

\end{document}
